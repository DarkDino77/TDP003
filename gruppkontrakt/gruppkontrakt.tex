\documentclass{mall}

\newcommand{\version}{Version 1.0}
\author{Klas Arvidsson, \url{klas.arvidsson@liu.se}\\
  Pontus Haglund, \url{pontus.haglund@liu.se}\\
  Emma Enocksson Svensson, \url{emma.enocksson@liu.se}\\
  Filip Strömbäck, \url{filip.stromback@liu.se}}
\title{Gruppkontrakt}
\date{2020-10-01}
\rhead{}


\begin{document}
\projectpage

% När ni har fyllt i dokumentet kan denna rubrik tas bort helt.




\section{Hur vi arbetar tillsammans}

% När ni har fyllt i dokumentet kan instruktionerna nedan tas bort.

Under mötet, utgå ifrån hur ni har kommit fram till att ni bäst stöttar alla i gruppen, och skriv ner hur ni bäst arbetar tillsammans för att kunna ge varandra det stöd ni
vill ge. Detta blir i slutändan gemensamma ''förhållningsregler'' för hur ni arbetar i gruppen, och
vad ni förväntar er av varandra.


\begin{itemize}
\item \textbf{Vilka tider arbetar vi, och vilka tider är vi nåbara utöver detta?}

  Dagar vi ska arbeteta är måndag till fredag.

  Vi ska arbeta mellan klockan 8 - 17. Med en timme till lunch.

  Inga tider utöver detta.

  I snitt ska vi arbeta 23 timmar per vecka.

\item \textbf{Hur kommunicerar vi med varandra? Vilka verktyg/kanaler använder vi? Hur och när är det okej att vi avbryter varandra?}

  Vi kommunicerar med varandra genom discord eller i person. Vi har också tillgång till varandras telfon nummer.
  
  Man väntar på att den andra personen pratar till punkt.

  Vi avbryter varandra när man känner att opponenten har fel. 

\item \textbf{Hur gör vi för att ge varandra möjlighet att framföra åsikter och tankar om uppgifter och idéer till arbetet?}

  Vi kommunicerar.

\item \textbf{Hur ofta tar vi paus? Ska vi hjälpas åt att påminna varandra om att ta paus?}

  Men man tar en paus när man känner att inte presterar bra nog. 

  Eller inte tar pausen om man känner att man är i arbetsläge.

\item \textbf{Arbetar vi tillsammans med uppgifter, eller var för sig?}

  Vi börjar med metod 1, kommer vi inte överrens metod 1 går vi över till metod 2. 

  Metod 1: Vi arbetar var för sig med överrenskommna uppgifter. Kalla till arbetspass om man har kört fast. 

  Metod 2: Vi arbetar i grupp genom hela projektet.

  Hela projektet kan delas in i 3 delar 

  Planering/architectur  Dessa behövs göras ihop då detta är högnivå beslut som båda behöver vara överrens om. Bra enligt boken del 1.(hitta ref att ta med reflektions dokumentet senare)
  
  -lofi prototyp
  
  -tidsplan 
  
  -projektplan 
  
  -datalager architectur

  Coding - separat kodning med sina delar från architecturen
  Efter vi har klad kådat i architectur delen. Delar vi upp projektet i två delar som vi separat codar med eventuel samman träff på de svårare delarna
  Efter man har skrivit klart en funktion ska man skriva relevanta kommentarer och dokumentering till funktionen.
  Kommentarer är en väldigt kort beskrivning av vad funktionen gör, samt möjligt en för klarning över hur den fungerar om det är icke trivial lösning
  Dokumentering ska vara en förklarning över hur man använder funktionen.
  Kanske kan vara värt att sicka kommentarer och Dokumentering till varandra för att kålla att vi har skrivit dom bra(om jag har gjort det). 


  Dokumentering
  Instalations manual skrivs i grupp då båda kommer behöva kunna visa behärskning över hur man gör detta vid muntan. Detta ska dubble kållas att det går på su datorer
  Systemdokumentation Vi har gjort architecturen ihop i planering stadiet, kanske behöver skrivas lite på hur man gör detta.
  Sedan tar vi och läser igenom varandras kommentarer och dokumentering för funktionerna.
  Sedna med minimal ändring(förhoppnings vis) stoppar in den i systemdokumentation.
  reflektions dokumentet skrivs induvidielt 

\item \textbf{Hur bestämmer vi vem som gör vad?}

  Veckomöten i person och minimöten.
  
  I veckomöten ska vi presentera hur långt man har kommit i sina arbetsuppgifter. Samt ska   

\item \textbf{Hur specifierar vi vad som ingår i varje uppgift, och när den är klar?}

  Skriv text här.

\item \textbf{Hur snabbt förväntar vi oss att en uppgift kan vara klar?}

  Så snabbt som möjligt. Hellre en tidig ledighet än crunch.

\item \textbf{Hur håller vi reda på att uppgifter vi identifierat inte glöms bort?}

  Skriv text här.

\end{itemize}




\end{document}
