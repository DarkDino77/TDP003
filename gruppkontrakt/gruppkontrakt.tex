\documentclass{mall}

\newcommand{\version}{Version 1.0}
\author{Klas Arvidsson, \url{klas.arvidsson@liu.se}\\
  Pontus Haglund, \url{pontus.haglund@liu.se}\\
  Emma Enocksson Svensson, \url{emma.enocksson@liu.se}\\
  Filip Strömbäck, \url{filip.stromback@liu.se}}
\title{Gruppkontrakt}
\date{2020-10-01}
\rhead{}


\begin{document}
\projectpage

% När ni har fyllt i dokumentet kan denna rubrik tas bort helt.




\section{Hur vi arbetar tillsammans}

% När ni har fyllt i dokumentet kan instruktionerna nedan tas bort.

Under mötet, utgå ifrån hur ni har kommit fram till att ni bäst stöttar alla i gruppen, och skriv ner hur ni bäst arbetar tillsammans för att kunna ge varandra det stöd ni
vill ge. Detta blir i slutändan gemensamma ''förhållningsregler'' för hur ni arbetar i gruppen, och
vad ni förväntar er av varandra.


\begin{itemize}
\item \textbf{Vilka tider arbetar vi, och vilka tider är vi nåbara utöver detta?}

  Dagar vi ska arbeteta är måndag till fredag.

  Vi ska arbeta mellan klockan 8 - 17. Med en timme till lunch.

  Inga tider utöver detta.

  I snitt ska vi arbeta 23 timmar per vecka.

\item \textbf{Hur kommunicerar vi med varandra? Vilka verktyg/kanaler använder vi? Hur och när är det okej att vi avbryter varandra?}

  Vi kommunicerar med varandra genom discord eller i person. Vi har också tillgång till varandras telfon nummer.
  
  Man väntar på att den andra personen pratar till punkt.

  Vi avbryter varandra när man känner att opponenten har fel. 

\item \textbf{Hur gör vi för att ge varandra möjlighet att framföra åsikter och tankar om uppgifter och idéer till arbetet?}

  Vi kommunicerar.

\item \textbf{Hur ofta tar vi paus? Ska vi hjälpas åt att påminna varandra om att ta paus?}

  Men man tar en paus när man känner att inte presterar bra nog. 

  Eller inte tar pausen om man känner att man är i arbetsläge.

\item \textbf{Arbetar vi tillsammans med uppgifter, eller var för sig?}

  Vi börjar med metod 1, kommer vi inte överrens metod 1 går vi över till metod 2. 

  Metod 1: Vi arbetar var för sig med överrenskommna uppgifter. Kalla till arbetspass om man har kört fast. 

  Metod 2: Vi arbetar i grupp genom hela projektet.

\item \textbf{Hur bestämmer vi vem som gör vad?}

  Veckomöten i person och minimöten.
  
  I veckomöten ska vi presentera hur långt man har kommit i sina arbetsuppgifter. Samt ska   

\item \textbf{Hur specifierar vi vad som ingår i varje uppgift, och när den är klar?}

  Skriv text här.

\item \textbf{Hur snabbt förväntar vi oss att en uppgift kan vara klar?}

  Så snabbt som möjligt. Hellre en tidig ledighet än crunch.

\item \textbf{Hur håller vi reda på att uppgifter vi identifierat inte glöms bort?}

  Skriv text här.

\end{itemize}




\end{document}
