\documentclass{mall}

\newcommand{\version}{Version 1.0}
\author{Dennis Abrikossov, \url{Denab905@student.liu.se}\\
Philip Bromander, \url{Phibr608@student.liu.se}}
\title{Gruppkontrakt}
\date{2023-09-04}
\rhead{}


\begin{document}
\projectpage

% När ni har fyllt i dokumentet kan denna rubrik tas bort helt.




\section{Hur vi arbetar tillsammans}

% När ni har fyllt i dokumentet kan instruktionerna nedan tas bort.

Under mötet, utgå ifrån hur ni har kommit fram till att ni bäst stöttar alla i gruppen, och skriv ner hur ni bäst arbetar tillsammans för att kunna ge varandra det stöd ni
vill ge. Detta blir i slutändan gemensamma ''förhållningsregler'' för hur ni arbetar i gruppen, och
vad ni förväntar er av varandra.


\begin{itemize}
\item \textbf{Vilka tider arbetar vi, och vilka tider är vi nåbara utöver detta?}

  Dagar vi ska arbeteta är måndag till fredag.

  Vi ska arbeta mellan klockan 8 - 17. Med en timme till lunch.

  Inga tider förväntas man arbeta utöver detta.

  I snitt ska vi arbeta 23 timmar per vecka.

\item \textbf{Hur kommunicerar vi med varandra? Vilka verktyg/ kanaler använder vi? Hur och när är det okej att vi avbryter varandra?}

  Vi kommunicerar med varandra genom Discord eller i person. Vi har också tillgång till varandras telefonnummer.
  
  Man väntar på att den andra personen pratar till punkt.

  Vi testar Overleaf för att skriva gemensamma LaTeX dokument.

  Vi använder GitLab för att dela projektfiler med varandra.

\item \textbf{Hur gör vi för att ge varandra möjlighet att framföra åsikter och tankar om uppgifter och idéer till arbetet?}

  Vi kommunicerar på möten och kommer fram till vad är bäst för projektet.

  Vi kan också nå varandra utanför möten eller arbetspass genom Discord för att upplysa om ändringar eller problem.

\item \textbf{Hur ofta tar vi paus? Ska vi hjälpas åt att påminna varandra om att ta paus?}

  Man tar en paus när man känner att inte presterar bra och man inte lyckas med sina uppgifter. Man kan också välja att inte ta en paus om man känner att man är i ett bra flyt.

\item \textbf{Arbetar vi tillsammans med uppgifter, eller var för sig?}

  Hela projektet kan delas in i 3 delar.

  \subsubsection{Planering/ arkitektur:}
  Detta måste göras i grupp då detta är ett beslut som påverkar hela projektet och som alla behöver vara överrens om.
  
  %Bra enligt boken del 1, hitta ref att ta med reflektions dokumentet senare.

  %- LoFi prototyp
  %- Tidsplan 
  %- Projektplan
  %- Datalager architectur

  \subsubsection{Programmering:}
  Individuell programmering efter uppgifter diligerade från veckomöten efter en kladdkod av arkitekturen är skapad. Vi delar upp projektet i många välfördelade arbetsuppgifter. Vi skriver kod enskilt efter sina fördelade uppgifter men om en större uppgift eller ett svårare problem uppstår kan man programmera tillsammans för att lösa problemet.

  %Efter varje session bör varje ny funktion ha en relevant kommentar och vid projektets slut skall alla funktioner ha kommentarer som kan föras in i slutliga dokumentationen. Dessutom skall mer avancerade funktioner och moduler ha extra dokumentation. Kommentarer skall innehålla minst en sammanfattning av vad funktionen tar in och returnerar samt vad den gör med datan och om funktionen kan felmeddela om det är applicerbart. Dokumentering ska vara en förklarning över hur man använder funktionen. Kan för små till mellanstora funktioner vara kommenteringen. Man upplyser varandra under möten om man ser en funktion med kommentar som inte håller upp till standard.

  \subsubsection{Dokumentering:}
  Installationsmanual skrivs i grupp då alla behöver kunna visa behärskning över hur man gör detta vid muntan. Systemdokumentationen skrivs enskilt för den del man ansvarar för. Därefter granskar man varandras dokumentation innan den förs in i gemensamma systemdokumentationen.
  
  Reflektionsdokumentet skrivs enskilt då detta skall vara en personlig utvärdering.

  %Vi skall också dubbelkolla att det går att köra på SU-datorerna! Systemdokumentationen kommer innehålla ett sekvensdiagram \iffalse flödesschema/ flödesdiagram \fi som kommer att härstamma ifrån projektplanen.

\item \textbf{Hur bestämmer vi vem som gör vad?}

  Under kodningsfasen har vi veckomöten tisdag och torsdag.

  Dagliga minimöten emellan där man checkar av hur långt man har kommit med sina arbetsuppgifter samt om det finns något problem man har fastnat på.
  
  I veckomöten ska vi presentera hur långt man har kommit i sina arbetsuppgifter. Man skall också distributera ut nya arbetsuppgifter.

  Under planeringfasen arbetar gruppen ihop med projektplanen vilket inkluderar tidsplanen och Low-Fidelity Prototypen.
  
  Individuella reflektionsdokumentet skriver man enskilt.

\item \textbf{Hur specifierar vi vad som ingår i varje uppgift, och när den är klar?}

  Vi delar upp arbetsuppgifter på veckomöten och vi bedömer hur lång tid varje uppgift kommer att ta var för sig. Vi kollar på minimöten hur långt vi har kommit och om det kan behövas mer tid eller dela upp en arbetsuppgift till mindre.

\item \textbf{Hur snabbt förväntar vi oss att en uppgift kan vara klar?}

  Vi förväntar oss att en uppgift blir färdig mellan veckomöten. Blir uppgifter inte färdig tills nästa veckomöte diskuteras varför och potentiella lösningar.

\item \textbf{Hur håller vi reda på att uppgifter vi identifierat inte glöms bort?}

  Vi håller ett LaTeX-dokument med pseudokod, moment som skall göras, moment som arbetas på, och moment som är färdiga. Allting är skrivet under varsin rubrik.

\end{itemize}

\end{document}
